Throughout the duration of this project, our conceptualization has undergone iterative refinement, integrating insights gleaned from collaborative discussions and the outcomes of prototyping endeavors. 
Initially, our project envisioned the development of a comprehensive terrain editor operating within three-dimensional space. 
However, subsequent deliberations with our instructors and internal team discussions led to the realization that the Haply 2DIY platform lacked the requisite capacity to accurately perceive depth within a three-dimensional environment. 
Consequently, we resolved to focus primarily on surface-level terrain features, directing our attention towards the painting of objects and textures while retaining the innovative capability to manipulate scale.

As a result of these findings, we concluded that the most judicious course of action entailed positioning our project as a complementary plugin within the Unity ecosystem, specifically tailored to augment the functionality of Unity's terrain game object. 
In essence, our endeavor reframes the editor as an extension of Unity's game engine, enhancing the user experience by facilitating the intuitive painting of objects and textures via the Haply 2DIY.

Amidst our prototyping endeavors, we encountered the unexpected ease of transitioning from haptic texture to haptic force feedback. 
Leveraging Unity's foundational concepts of textures and colliders, we observed that altering the scale of the End-Effector representation sphere significantly influenced its collision behavior with surrounding objects. 
At certain scales, the End-Effector exhibited the ability to navigate on top of obstacles that previously blocked its progress.

Despite the current rudimentary state of our project, we believe it effectively showcases the future possibilities granted by such integration.
Especially providing the encouraging results we received from our user evaluation and testing.
Moreover, the extensibility of the current codebase utilizing Unity's robust scripting systems, which lay a solid foundation for future expansion. 
For instance, there are multiple ways to bring enhancements to the projects, one of which would come in form of improvements and additional functionalities to the painting UI such as being able to change the brush radius.
Additionnaly, experimentation is required to counteract our current limitations and reinforce our texture haptic feedback.