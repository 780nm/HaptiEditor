Throughout the development of this project, our concept has undergone iterative refinement, integrating insights gleaned from collaborative discussions and the outcomes of prototyping endeavors. 
Initially, we targeted terrain editing within three-dimensional space. 
However, subsequent deliberations with our instructors and internal team discussions led to the realization that the Haply 2DIY platform would fail to accurately perceive depth within a three-dimensional environment, yielding poor outcomes. 
Consequently, we resolved to focus primarily on surface-level terrain features, directing our attention towards the painting of objects and textures on a flat surface while retaining the innovative capability to manipulate scale.

As a result of these findings, we positioned our project as a complementary plugin within the Unity ecosystem, specifically tailored to augment the functionality of Unity's terrain game object. 
Our endeavor reframes the editor as an extension of Unity's game engine, enhancing the user experience by facilitating the intuitive painting of objects and textures via the Haply 2DIY.

In our prototyping endeavors, we encountered the unexpected ease of transitioning from haptic texture to haptic force feedback. 
Leveraging Unity's foundational concepts of textures and colliders, we observed that altering the scale of the End-Effector representation sphere significantly influenced its collision behavior with surrounding objects. 
At certain scales, the End-Effector exhibited the ability to navigate on top of obstacles that previously blocked its progress.

As a proof-of-concept, we believe our work effectively showcases the future possibilities granted by haptic integration in a game development context; a sentiment supported by the encouraging results we received from our user evaluation and testing.
Moreover, we present our project as proof of the extensibility offered by using Unity's robust scripting systems, which lays a solid foundation for future expansion. 
For instance, future work may bring improvements and additional functionalities to the painting UI such as being able to change the brush radius. Experimentation is required to counteract our current limitations and improve the utility of texture-driven haptic feedback to users of the project.